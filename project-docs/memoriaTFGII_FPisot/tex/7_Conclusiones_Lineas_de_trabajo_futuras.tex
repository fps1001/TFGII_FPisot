\capitulo{7}{Conclusiones y Líneas de trabajo futuras}
\section*{Conclusiones}
\section*{Líneas de trabajo futuras}
Este TFG termina con la entrega de una versión funcional completa y estable para poder ser utilizada por usuarios que quieran planificar rutas turísticas sostenibles. Durante el desarrollo de este TFG han surgido muchas líneas de trabajo futuro que pueden servir cómo punto de partida en el caso de ponerse en explotación. A continuación se enuncian algunas líneas que pueden ayudar a definir una futura evolución funcional.
\begin{itemize}
    \item \textbf{Gamificación:} Incluir recompensas por rutas completadas o distancia recorrida con un medio ecológico. Logros que desbloqueen aspectos visuales del icono de la aplicación.
    \item \textbf{Ratings:} Publicar y valorar las tours generados permitiendo la búsqueda de los mismos a otros usuarios.
    \item \textbf{Planificador de rutas:} Añadir otras fuentes de datos que optimicen la sostenibilidad de los tours, por ejemplo, utilizando información por satélite para definir rutas basadas en áreas con mayor cobertura de sombra.
    \item \textbf{Multiplataforma:} La aplicación podría beneficiarse de su adaptación a otras plataformas, donde se tendría que tener en cuenta principalmente los permisos de localización. Al utilizar Flutter esta adaptación se podría realizar sobre el mismo código base, facilitando en gran medida su consecución. De hecho se trató de conseguir desde un principio en el desarrollo de este \acrlong{tfg} su adaptación a iOS configurando los permisos de localización, sin embargo para probar estas funcionalidades es necesario contar con un equipo Mac. Legalmente, no es posible emular macOS en máquinas virtuales en otras plataformas, y tampoco se disponía de un equipo Mac propio ni de acceso a uno en préstamo. 
	\item \textbf{Reconocimiento de lugares (Landmark Recognition):} Firebase ofrece una función avanzada de reconocimiento de lugares~\cite{firebase_mlkit_landmarks} que permitiría añadir una capa de personalización en la aplicación. Esta funcionalidad permitiría al usuario tomar una foto de uno de los \acrlong{pdi} visitados, y la aplicación podría identificar automáticamente el lugar y actualizar el estado de la ruta en tiempo real, indicando que se ha completado la visita. Esta mejora no solo facilitaría el seguimiento de la ruta, sino que también proporcionaría una experiencia de usuario más interactiva y dinámica.
	\item \textbf{Registro y autenticación de usuarios:} Para implementar algunas de las mejoras propuestas, es necesaria una gestión integral de usuarios. Actualmente, no existe un proceso de registro o autenticación, ya que es Cloud Firestore quien asigna un identificador único a cada usuario. Aunque esta gestión permite identificar los \textit{Eco City Tours} generados por cada usuario, presenta limitaciones, como la pérdida de acceso a los tours al reinstalar la aplicación, debido a la asignación de un nuevo identificador. Además, los tours previos quedarían en la base de datos sin posibilidad de ser recuperados. Implementar un sistema de registro y autenticación una vez configurado Firebase sería una tarea relativamente sencilla, que incluiría generar una pantalla de registro y acceso, así como modificar la configuración de Cloud Firestore para incluir módulos de autenticación mediante servicios como correo electrónico o cuentas de Google.
\end{itemize}