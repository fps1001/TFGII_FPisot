\capitulo{5}{Aspectos relevantes del desarrollo del proyecto}

Durante el desarrollo de un \acrlong{tfg}, es inevitable tomar decisiones que impactan de manera significativa en el resultado final. En este capítulo se describen las decisiones clave que han influido en la evolución y estado final de Eco City Tours. La justificación de estas decisiones se espera que pueda servir como una referencia útil para futuros compañeros o desarrolladores que enfrenten objetivos similares. 

\section{Personalización de LLM}
A la hora de obtener información que procesar por el método \acrshort{rag} se valoraron muchas fuentes de datos. Uno de los origenes de datos de información turística favoritos de los usuarios es Tripadvisor. Contar con la información actualizada de este gigante turístico suponía un gran aliciente. Sin embargo se desestimó su uso por varias razones: la información se podía obtener a través del método scraping o webscraping que toma la información en bruto de la página web y se podía postprocesar. Dicha práctica incumpliría los Términos de Uso del Servicio, ya que Tripadvisor usa un acceso a través de API para obtener la información de su base de datos. En primer lugar se necesita una forma de pago para poder empezar a usar el servicio y su utilización si sobrepasa las 5.000 peticiones al mes incurriría en gastos al desarrollador. En este caso se buscaron alternativas que funcionaran de manera análoga a Tripadvisor para nutrir el \acrshort{rag}. \todo{INDICAR CUÁLES CUANDO SE HAGA}

\section{Elección de servicios Google sobre tecnología \acrlong{osm}}

Desde un comienzo en el proyecto se quería utilizar \textbf{código abierto}, pues su filosofía se alinea mejor con los valores aprendidos en la Universidad, donde se promociona el uso de herramientas que no supongan un coste para el alumno, se fomenta su uso evitando la posible discriminación económica y una forma de trabajar colaborativa.

De manera renuente se decide cambiar los servicios necesarios para la visualización y gestión de marcadores y rutas a los establecidos por Google. Los motivos que propiciaron este gran cambio fueron los siguientes:
\begin{itemize}
	\item \textbf{Soporte de un gigante tecnológico}: las herramientas de código abierto aunque algunas tienen un gran seguimiento por la comunidad no pueden competir con la documentación, ejemplos de desarrolladores y tecnología de uso de una potencia como Google.
	\item \textbf{Integración}: Flutter forma parte de Google, lo que supone una integración nativa que hace de su funcionamiento y robustez una de las herramientas usadas.
	\item \textbf{Riesgos asociados a complementos de terceros}: Durante el desarrollo, se exploraron soluciones como \acrfull{osrm} para gestionar rutas, pero estas requerían procesar manualmente las conexiones con el servicio o confiar en paquetes de terceros. Estos paquetes, aunque facilitan el desarrollo, presentan un riesgo mayor debido a su posible discontinuidad o incompatibilidad con futuras actualizaciones. En cambio, el uso de herramientas como Dio \cite{dio_package}, un paquete marcado como favorito por Flutter, garantiza un soporte nativo y más estable en el ecosistema de Google.
\end{itemize} 

\section{Elección de google generative ai como primer LLM}
Durante el desarrollo se estudiaron múltiples opciones a la hora de obtener la información desde un \acrshort{llm}. 
Siguiendo con el ecosistema de Google se probó con éxito el modelo \textit{gemini-1.5-flash} gracias al paquete \textit{google\_generative\_ai} que permite configurar el modelo desde su página web para más tarde exportar la configuración a un archivo Dart. Sus resultados cambiaban con los gustos del usuario, y después de aumentar la precisión de las coordenadas GPS éstas eran correctas.
\
\section{Descarte de módelo en local}
Aunque en un principio el ser capaz de probar la aplicación de una manera 
\todo{Continuar...}

\section{Elección de Google Places}
El principal inconveniente del modelo liviano de Gemini \cite{deepmindgemini} fue que las URL que marcaba como imágenes no eran accesibles desde el mapa. Para resolver este problema, se consideró utilizar otros servicios de obtención de imágenes en Internet, como Unsplash \cite{unsplash}, o incluso la API de búsqueda de Google. Sin embargo, en términos de coste, resultaba preocupante que estas alternativas incurrieran en gastos tras superar el límite de 100 peticiones diarias. Finalmente, se descartaron estas opciones y se adoptó como solución el servicio de Google Places, que utiliza la misma clave API que se emplea para la gestión de mapas. Este servicio, además de proporcionar más información sobre los \acrlong{pdi}, ofrece una imagen fiable que se puede almacenar en caché gracias al paquete \textit{cached\_network\_image}.

\section{Aclaración sobre el coste de Servicios Google}
Para la versión inicial de la aplicación de este \acrshort{tfg}, se optó por utilizar MapBox \cite{mapbox}, una plataforma de mapas con características similares a las de Google. Sin embargo, a medida que avanzaba el desarrollo, se fueron integrando cada vez más servicios de Google, hasta el punto de reemplazar el servicio de optimización de rutas por el de Google. En cada cambio se evaluaba cuidadosamente si el número de peticiones mensuales superaría el límite que generaría costos adicionales para el desarrollador. No obstante, tras estimar que el número de peticiones estaría muy por debajo del umbral que implicaría un coste, se decidió aprovechar las ventajas de trabajar dentro de un mismo ecosistema. Como se muestra en la imagen \ref{fig:trafico_google} las peticiones de los diferentes servicios distan mucho de las por ejemplo 20.000 peticiones que serían 100 dolares, la mitad del dinero que da gratis Google durante el desarrollo de sus aplicaciones.
  \imagen{trafico_google}{Estadísticas de solicitudes de servicios}{1}
