\capitulo{2}{Objetivos del proyecto}

Este apartado explica de forma precisa y concisa cuáles son los objetivos que se persiguen con la realización del proyecto. Se puede distinguir entre los objetivos marcados por los requisitos del software a construir y los objetivos de carácter técnico que plantea a la hora de llevar a la práctica el proyecto.

\section{Objetivos Funcionales}

Los objetivos funcionales definen qué es lo que debe hacer el software desde la perspectiva del usuario final. Estos objetivos se centran en las funcionalidades y características que debe tener la aplicación para satisfacer las necesidades y expectativas de los usuarios. A continuación se detallan los objetivos funcionales del proyecto:

\begin{itemize}
    \item \textbf{Propuesta de Rutas Turísticas Personalizadas}: La aplicación debe ser capaz de generar rutas turísticas personalizadas basadas en las preferencias del usuario utilizando modelos de lenguaje de gran escala (\acrfull{llm}) y el framework \textbf{LangChain}.
    \item \textbf{Integración de Puntos de Interés (\acrfull{poi})}: La aplicación debe identificar y conectar diversos puntos de interés, proporcionando información relevante sobre cada uno de ellos.
    \item \textbf{Visualización de Rutas en Mapa}: La aplicación debe mostrar las rutas sugeridas en un mapa utilizando herramientas open-source como \acrfull{osm}.
    \item \textbf{Optimización para Ciclistas y Peatones}: La aplicación debe promover la movilidad sostenible sugiriendo rutas optimizadas para ciclistas y peatones.
    \item \textbf{Interfaz Intuitiva y Amigable}: La aplicación debe tener una interfaz de usuario intuitiva y fácil de usar que permita a los usuarios interactuar con las diferentes funcionalidades de manera eficiente.
\end{itemize}

\section{Objetivos Técnicos}

Los objetivos técnicos se refieren a los desafíos y metas técnicas que se deben abordar para desarrollar el software. Estos objetivos abarcan aspectos como la arquitectura del sistema, las tecnologías a utilizar y las metodologías de desarrollo. A continuación se detallan los objetivos técnicos del proyecto:

\begin{itemize}
    \item \textbf{Desarrollo con Flutter}: Utilizar el framework \textbf{Flutter} para el desarrollo de la aplicación móvil, aprovechando su capacidad para crear aplicaciones nativas multiplataforma con una sola base de código.
    \item \textbf{Implementación de \acrfull{llm} y \textbf{LangChain}}: Integrar modelos de lenguaje de gran escala (\acrshort{llm}) y el framework \textbf{LangChain} para la generación de rutas y procesamiento de información relevante.
    \item \textbf{Uso de Herramientas Open-Source}: Emplear herramientas y bibliotecas open-source como \acrfull{osm} para la visualización de mapas y rutas.
    \item \textbf{Optimización de Rendimiento}: Asegurar que la aplicación funcione de manera eficiente en diferentes dispositivos móviles, optimizando el rendimiento y el consumo de recursos.
    \item \textbf{Seguridad y Privacidad}: Implementar medidas de seguridad adecuadas para proteger los datos de los usuarios y garantizar la privacidad de la información personal.
    \item \textbf{Pruebas y Validación}: Realizar pruebas exhaustivas de la aplicación para asegurar su correcto funcionamiento, identificando y solucionando posibles errores antes de su lanzamiento.
\end{itemize}
