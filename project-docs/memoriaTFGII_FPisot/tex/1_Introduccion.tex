\capitulo{1}{Introducción}


El crecimiento de población en las ciudades CITA, el turismo como catalizador de la gentrificación supone el gran caballo de batalla de los ayuntamientos y gobiernos locales que han visto como la falta de legislación y control del turismo supone un grave problema agravado aún más si cabe por el malestar de las comunidades locales. La deriva de este problema de difícil solución se centra en una necesidad de desarrollar ciudades más sostenibles e inclusivas para evitar este malestar de las comunidades y su desarrollo incontrolado. A pesar de los avances en la promoción de un nuevo modelo, muchas urbes aún enfrentan desafíos significativos en la integración de prácticas sostenibles en la vida cotidiana de sus habitantes. La falta de información accesible y personalizada sobre rutas y actividades que promuevan la movilidad sostenible y el turismo responsable es un marco común que se debe desarrollar si se quiere evitar que el conflicto crezca sin fin. Esta brecha de información impide que tanto residentes como turistas adopten hábitos más sostenibles que beneficien a la comunidad local y al medio ambiente en un marco global.

A partir del año 2000, muchos gobiernos empezaron a desarrollar soluciones TIC que gracias a fondos económicos europeos promovieron el termino \acrfull{sdg11}, donde marcos de trabajo intentaban fomentar 

En este contexto, la aplicación móvil que proponemos se alinea con estos esfuerzos al proporcionar una herramienta práctica y accesible para la promoción del \acrfull{ods11} y la movilidad sostenible.

Este \acrfull{ods}, se centra en hacer que las ciudades y los asentamientos humanos sean sostenibles y emerge como un pilar fundamental de desarrollo si la población en los centros de las ciudades sigue creciendo. Según el informe de síntesis de la UNESCO sobre el ODS 11, este objetivo no solo es crucial por sí mismo, sino que actúa como un multiplicador, influyendo indirectamente en la consecución de otros \acrshort{ods} debido a su enfoque integral y transversal\footnote{\href{https://uis.unesco.org/sites/default/files/documents/sdg_11_synthesis_report_2023_v11_0_4.pdf}{UNESCO. (2023). SDG 11 Synthesis Report.}}.

La aplicación producto de este \acrshort{tfg}, desarrollada con Flutter, utiliza modelos de lenguaje de gran escala (\acrfull{llm}) y el marco de trabajo LangChain para generar rutas turísticas personalizadas. Estas rutas conectan puntos de interés (\acrfull{poi}) y se visualizan mediante herramientas open-source como \acrfull{osm}.

La aplicación se enfoca en las preferencias del usuario, ofreciendo rutas optimizadas para ciclistas y peatones, y promoviendo así la movilidad sostenible. Al integrar datos y tecnología avanzada, nuestra solución no solo facilita una experiencia turística enriquecedora, sino que también fomenta prácticas sostenibles que pueden tener un impacto positivo en la comunidad y el medio ambiente.

De esta manera se favorece el desarrollo sostenible de las ciudades..........

\end{document}