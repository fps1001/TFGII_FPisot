\apendice{Plan de Proyecto Software}

\section{Introducción}
El Plan de Proyecto Software es el documento clave que dirige el proceso de desarrollo de la aplicación móvil creada. Este apéndice tiene como objetivo detallar los aspectos críticos de la planificación y gestión del proyecto, asegurando una implementación eficiente y efectiva.
La planificación temporal del proyecto se ha llevado a cabo con el \emph{uso de la metodología ágil} buscando dividir el desarollo en tareas, sprints e hitos que producen un resultado iterativos y bien estructurado, lo que conlleva a una mayor flexibilidad y a la adaptación más efectiva frente los cambios. 

A continuación, se determinará la viabilidad, que reflejará los \emph{recursos humanos y materiales}, así como los costes asociados necesarios para su valoración. La viabilidad incluirá una estimación de los fondos que se basarán en el salario de un trabajador de la imaginación, así como un análisis de los posibles riesgos y su mitigación. Los aspectos económicos y técnicos de la viabilidad son fundamentales, ya que de ellos depende de que el proyecto esté en los límites establecidos y cumpla con los objetivos propuestos.

Este plan es esencial para la gestión del proyecto ya que sirve como una guía detallada, ayudando así a identificar y mitigar riesgos así como a la utilización eficaz de los recursos. Con el enfoque estructurado y ágil, proporcionado por este plan, el equipo de desarrollo podría entregar un producto de alta calidad que, además, cumplirá con el nivel de satisfacción alcanzado entre los usuarios o clientes. 
\section{Planificación temporal}
 Como se ha mencionado anteriormente, la planificación temporal del proyecto se ha llevado a cabo con el uso de metodología ágil: se basa en la división del desarrollo en tareas, sprints e hitos que producen un resultado iterativo y bien estructurado. Esto conlleva una mayor flexibilidad y a la adaptación más efectiva frente a los cambios.

 Algunas herramientas utilizadas para la planificación temporal han sido GitHub y Zube. Ésta última ha permitido la organización de las tareas en tableros Kanban. El uso de GitHub ha permitido gestionar un control de versiones.

 A continuación veremos como la planificación temporal se ha llevado a cabo en diferentes sprints, cómo se ha ido iterando en las diferentes partes del proyecto y cómo se han ido cumpliendo los hitos propuestos. Para ello se mostrarán diferentes diagramas basadas en métricas ágiles.

 Cada tarea se ha dividido en diferentes historias de usuario, que se han ido completando en cada sprint. Cada sprint ha tenido una duración de una o dos semanas, y se han ido completando las tareas propuestas en cada uno de ellos.
 Un ejemplo se puede observar en la siguiente figura ~\ref{fig:issue12}
 \imagen{issue12}{Tarea 12 mostrada en GitHub con la descripción, hito y etiquetas de la tarea a realizar.}

Gracias al uso de la herramienta Zube, se ha podido llevar un control de las tareas a realizar, las tareas completadas y las tareas pendientes. Además, se ha podido llevar un control de los hitos propuestos y de las historias de usuario completadas en cada sprint. Un ejemplo de ello se puede observar en la siguiente figura ~\ref{fig:sprint1}
\imagen{sprint1}{Tablero Kanban de Zube con la gestión de tareas del Sprint 1.}

\subsection{\textit{Hitos}}
Los hitos o milestones son puntos de referencia que marcan el final de un conjunto de tareas. En este proyecto se han definido los siguientes hitos:

\begin{itemize}

    \item \textbf{Kick-off} Completado el 30 de julio de 2024. Puesta en marcha del proyecto. A partir de las reuniones mantenidas con el tutor, se necesita tener todas las herramientas preparadas para empezar a desarrollar tanto la aplicación como su documentación.
    
    \item \textbf{MPV - Mínimo Producto Viable} Completado el 2 de septiembre de 2024. Se define el MVP como una aplicación móvil que sobre un mapa OSM muestre la ubicación de usuario, obtenga unos \acrshort{pdi} básicos y una ruta que las una.

    \item \textbf{Checkpoint 1 de documentación} Completado el 2 de septiembre de 2024. Este milestone agrupa las tareas relacionadas con la creación y actualización de la documentación del proyecto hasta la reunión con el tutor el 1 de septiembre de 2024.

    El objetivo es tener una documentación suficiente para que el tutor pueda dar feedback acerca de la misma y poder corregir errores.

    \item \textbf{Prototipo con tours generados por LLM} Completado el 1 de octubre de 2024. El objetivo es transitar desde una aplicación inicial capaz de mostrar lugares y rutas en un mapa, hacia una aplicación que sea capaz de conseguir que estos mismos marcadores y polilíneas sean generados a través de un LLM. \label{hito:prototipo_llm}
    
    \item \textbf{Prototipo Prompting} Completado el 15 de octubre de 2024. Este prototipo se puede realizar en un cuaderno Jupyter y su objetivo es mostrar la evolución en el prompt que dará como resultado unos \acrshort{pdi} de mayor calidad.
    
    \item \textbf{Desarrollo de aplicación completo} Completado el 12 de noviembre de 2024. Se consideran todas las funcionalidades que debe tener la aplicación a presentar como completadas.
    
\end{itemize}

\subsection{Organización en \textit{Sprints}}

Al comenzar este proyecto durante periodo no lectivo se realizaron los Sprint con variación de tiempo de una o dos semanas en función de la planificación personal. Una vez comenzado el curso y con la ayuda del tutor se realizaron reuniones que han servido para, siguiendo la metodología \textit{Agile}, revisar el Sprint anterior, planificar el siguiente y hacer una pequeña retrospectiva para mejorar el trabajo conjunto.


\begin{itemize}
    \item \textbf{Sprint 0 - Kick-off(22/07/2024 - 29/07/2024):} Después de las reuniones con el tutor, se establecen los objetivos del proyecto y se comienza a trabajar en la puesta en marcha del proyecto. Se establecen las herramientas a utilizar y se comienza a trabajar en la documentación del proyecto. 33 puntos de historia en 5 tareas.
    
    ~\ref{fig:bd-kick-off sprint}
    \imagen{bd-kick-off sprint}{Figura burndown del Sprint Kick-off.}

    \item \textbf{Sprint 1 - Investigación LLM y desarrollo básico de aplicación con mapa (29/07/2024 - 05/08/2024):} Con las herramientas y una idea previa establecida, es el momento de empezar a desarrollar.
    

    Objetivos: seguir formándome en LLM y las opciones que pueda implementar en el prototipo de prompt.
    Empezar a desarrollar la aplicación móvil con las características básicas.
    Aprender a documentar sprints, indicar elementos que tendré que documentar y aquellos que tenga claro ir documentando para hacer un avance significativo que pueda evaluar mi tutor.
    ~\ref{fig:bd-s1}
    \imagen{bd-s1}{Figura burndown del Sprint 1: Investigación LLM y desarrollo básico de aplicación con mapa.}
    
    \item \textbf{Sprint 2 - Implementación solución GIS (05/08/2024 - 12/08/2024):} A partir del concepto básico, se añaden pequeñas mejoras en los tres aspectos del proyecto.
    
    Objetivos: mejorar el prototipo de prompting del cuaderno Jupyter hasta incorporar un sistema RAG, incluir marcadores al mapa en cuanto al desarrollo y continuar con la documentación.
    ~\ref{fig:bd-s2}
    \imagen{bd-s2}{Figura burndown del Sprint 2: Implementación solución GIS.}
    
    \textit{Dificultades encontradas}: la documentación me hizo perder mucho tiempo debido a problemas con las librerías, después de mucho tiempo reinicié el proyecto desde la plantilla dada, insertando el texto, lo que solucionó el problema. En cuanto al diseño de la aplicación, el desarrollo fue lento al tener que evaluar diferentes opciones ya que la mayoría de fuentes utilizan mapas de Google, opción que se quería descartar.
 

    \item \textbf{Sprint 3 - MPV(12/08/2024 - 22/08/2024):} Este sprint fue más largo que los anteriores para mejorar el resultado final ya que la intención era dejar el proyecto en un estado de revisión lo más completo posible para afrontar la reunión prevista para inicio de septiembre con el tutor del mismo. Al intentar desarrollar la tecnología de enrutado del usuario se comprendió lo que ya se intuía en el sprint anterior y es que basar el trabajo en servicios de Google iba a reportar en un desarrollo más fácil y un resultado más robusto y fiable como se justifica en la sección 5 de la memoria de este \acrshort{tfg}.
~\ref{fig:bd-s3}
\imagen{bd-s3}{Figura burndown del Sprint 3: MPV.}
    
    \item \textbf{Sprint 4 - Servicios MapBox y LLM, preparación reunión inicio curso.(22/08/2024 - 02/09/2024):} El objetivo es mostrar la versión más completa de la aplicación, la documentación y ahondar en el uso de nuevas herramientas como Figma como herramienta de diseño de aplicaciones y LangFlow a la hora de utilizar otro modelo de prototipo.
    ~\ref{fig:bd-s4}
    \imagen{bd-s4}{Figura burndown del Sprint 4: Servicios MapBox y LLM, preparación reunión inicio curso.}
    
\item \textbf{Sprint 5 - Aplicación con origen de datos LLM (Preparación y documentación) (04/09/2024 - 14/09/2024):} Después de la reunión con el tutor de inicio de septiembre y habiendo cumplido los objetivos de los primeros hitos se decide continuar intentando alcanzar el hito \ref{hito:prototipo_llm}. Para ello se prepara y documenta primero en este sprint el desarrollo necesario.
~\ref{fig:bd-s5}
\imagen{bd-s5}{Figura burndown del Sprint 5: Aplicación con origen de datos LLM (Preparación y documentación).}

\item \textbf{Sprint 6 - Desarrollo y Finalización prototipo google\_generative\_ai como LLM (18/09/2024-01/10/2024):} Habiendo encontrado una solución óptima al modelo LLM a utilizar se propone realizar un prototipo que implemente la interfaz de usuario y su conexión con el modelo LLM. En los diferentes apartados encontramos. Se trata de un sprint de prominente desarrollo de la aplicación. Se consiguió reestructurar todo el código centralizando labores de gestión del tour generado y sus \acrshort{pdi} manteniendo la modularidad del código.
~\ref{fig:bd-s6}
\imagen{bd-s6}{Figura burndown del Sprint 6:  Desarrollo y Finalización prototipo google\_generative\_ai como LLM.}
	
\item \textbf{S7 - Consolidación y Calidad (02/10/2024 - 22/10/2024)} Habiendo cumplido el hito \ref{hito:prototipo_llm} y teniendo una aplicación con muchas funcionalidades buscadas, durante este sprint se busca consolidar el código y dotarlo de una calidad y mantenimiento con herramientas de soporte como Sonar Cloud o Logger.
~\ref{fig:bd-s7}
\imagen{bd-s7}{Figura burndown del Sprint 7: Consolidación y Calidad.}

\item \textbf{S8 - Cobertura de tests y guardado de rutas (23/10/2024 - 12/11/2024):} Se procede a implementar las últimas funcionalidades del desarrollo de la aplicación. Además, se busca que la integración con sonarcloud confirme que se trabaja con un standard de calidad para lo que se necesita la cobertura de tests y por último se retoma el trabajo de documentación. 
\imagen{bd-s8}{Figura burndown del Sprint 8: Cobertura de tests y guardado de rutas.}

\item \textbf{S9 - Testing y selección de rol (13/11/2024 - 03/12/2024):} Se estudia la posibilidad de generar diferentes asistentes dandole al modelo distintos roles. Se pretende realizar el test que llevará la aplicación en este sprint. Además, tendrá se realizará el inicio de apartados de anexos en cuanto a documentación. 
\imagen{bd-s8}{Figura burndown del Sprint 9: Testing y selección de rol.}

\end{itemize}

\subsection{Métricas Ágiles}
Gracias al uso de Zube se emplean fácilmente diferentes métricas ágiles que han sido vitales para la evaluación del desarrollo de la aplicación y su organización a través de los sprints citados. Algunos de los artefactos usados son los siguientes:
\subsubsection{Gráficos burnup / burndown} 
Muestran a lo largo del tiempo de un sprint la evolución de tareas realizadas por el equipo de desarrollo. En la explicación de los sprints se pueden ver los gráficos asociados al mismo
\imagen{burnup-s6}{Gráfico Burnup del Sprint 6}{1}
\subsubsection{Gráfico de velocidad} 
Permite comprobar el trabajo realizado en los diferentes sprints de manera que resulte lo más constante posible.


\section{Estudio de viabilidad}
En esta sección se analizan los aspectos clave para determinar si la implementación del proyecto y el desarrollo de la aplicación móvil \textit{Eco City Tours} es viable desde un punto de vista legal y económico. Se destacan los valores positivos de la aplicación, como su independencia de intereses particulares y su enfoque sostenible, que la convierten en una propuesta valiosa en el mercado del fomento del turismo sostenible. \textbf{Se evaluará si los costos de desarrollo, el mantenimiento de los servicios y los posibles problemas legales se compensan con los beneficios esperados, determinando así su viabilidad.}
\subsection{Viabilidad económica}
\subsubsection{Costes de personal}
El desarrollo de la aplicación se realiza con un grupo de trabajo de un solo empleado de categoría profesional equivalente a la de un Ingeniero Informático. Según el XVIII Convenio Colectivo Estatal de Empresas de Consultoría, Tecnologías de la Información y Estudios de Mercado y de la Opinión Pública, publicado en el BOE el 26 de julio de 2023 \cite{boe2023_consultoria} podría estar entorno a los 2.000 € brutos
\subsubsection{Coste de hardware}
Para calcular el coste de los equipos se supone un período de amortización de 3 años para todo el hardware. Para realizar la aplicación el Ingeniero Informático usará un ordenador de sobremesa con procesador i7 y 16 Gb de memoria RAM capaz de utilizar los programas necesarios para el desarrollo de manera fluida. Periféricos para dicho equipo y un teléfono móvil Android para pruebas serían también necesarios. Por tanto y a modo resumen tenemos:
\tablaSmallSinColores
{Gastos de hardware}

\begin{table}[H]
	\centering
	\begin{tabular}{l r}
		\toprule
		Elemento & Costo (€) \\
		\midrule
		Ordenador & 800 \\
		Móvil & 500 \\
		Periféricos & 150 \\
		Conexión a Internet & 60/mes \\
		\bottomrule
	\end{tabular}
	\caption{Gastos de hardware}
	\label{tabla:gastos_hardware}
\end{table}





	
\subsubsection{Coste de software}

\subsubsection{Coste de los Servicios Google}
\subsubsection{Conclusiones basados en posibles beneficios}
Una vez analizado los costes devengados de la creación de la aplicación el beneficio de su explotación debe ser mayor al gasto ocasionado por su creación. Además, una vez publicada la aplicación en la \textit{Play Store} el beneficio obtenido por su explotación debe compensar el tiempo requerido por el personal en horas de mantenimiento de los posibles problemas que pueda experimentar como por ejemplo problemas por actualización de los componentes claves como modelos LLM, que puedan quedar obsoletos. El beneficio de la aplicación puede provenir de varias fuentes en función del modelo de explotación que se eligiese:
\begin{itemize}
	\item{Ingresos por descarga:} si la aplicación tiene un precio por descarga los ingresos serían el dinero obtenido menos una comisión por cada venta cobrada por Google de un 15\%. \cite{googleplay_commission}.
	Un precio de lanzamiento razonable podría ser no superior a los 4 euros, donde se obtendría aproximadamente 3,40 euros como beneficio. Tras un periodo de ajuste en función de las descargas, el precio por la misma se podría ajustar.
	\item{Publicidad:} empresas o entidades colaboradoras podrían financiar el desarrollo con publicidad siempre que ésta no suponga una perdida de independencia en los resultados de los modelos, fomente el turismo ecológico y no sea invasiva ni perjudique la experiencia de usuario.
	
\end{itemize}
\subsection{Viabilidad legal}


