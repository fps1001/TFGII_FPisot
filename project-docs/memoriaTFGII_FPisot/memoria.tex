\documentclass[a4paper,12pt,twoside]{memoir}

% Castellano
\usepackage[spanish,es-tabla]{babel}
\selectlanguage{spanish}
\usepackage[utf8]{inputenc}
\usepackage[T1]{fontenc}
\usepackage{lmodern} % Scalable font
\usepackage{microtype}
\usepackage{placeins}




\RequirePackage{booktabs}
\RequirePackage[table]{xcolor}
\RequirePackage{xtab}
\RequirePackage{multirow}

% Links
\PassOptionsToPackage{hyphens}{url}\usepackage[colorlinks]{hyperref}
\hypersetup{
	allcolors = {red}
}

% Acrónimos
\usepackage[acronym]{glossaries}
\makenoidxglossaries

\newacronym{llm}{LLM}{\textit{Modelos de Lenguaje a Gran Escala}}
\newacronym{ods}{ODS}{\textit{Objetivos de Desarrollo Sostenible}}
\newacronym{rag}{RAG}{\textit{Retrieval Augmented Generation}}

\newacronym{sdg}{SDG}{\textit{Sustainable Development Goals}}
\newacronym{sc}{Smart City}{\textit{Ciudades Inteligentes}}
\newacronym{sdg11}{SDG11}{\textit{Sustainable Cities and Communities}}
\newacronym{pdi}{PDI}{\textit{Puntos de Interés}}
\newacronym{tfg}{TFG}{\textit{Trabajo de Fin de Grado}}
\newacronym{nlp}{NLP}{\textit{Procesamiento del Lenguaje Natural}}
\newacronym{gis}{GIS}{\textit{Sistemas de Información Geográfica}}
\newacronym{cod}{CoT}{\textit{Cadena de pensamiento}}


% Acrónimos en inglés
\newacronym{llme}{LLM}{\textit{Large Language Models}}
\newacronym{sig}{SIG}{\textit{Geographic Information Systems}}
\newacronym{poie}{POI}{\textit{Points of Interest}}



% Ecuaciones
\usepackage{amsmath}

% Rutas de fichero / paquete
\newcommand{\ruta}[1]{{\sffamily #1}}

% Párrafos
\nonzeroparskip

% Huérfanas y viudas
\widowpenalty100000
\clubpenalty100000

% Imágenes

% Comando para insertar una imagen en un lugar concreto.
% Los parámetros son:
% 1 --> Ruta absoluta/relativa de la figura
% 2 --> Texto a pie de figura
% 3 --> Tamaño en tanto por uno relativo al ancho de página
\usepackage{graphicx}
\newcommand{\imagen}[3]{
	\begin{figure}[!h]
		\centering
		\includegraphics[width=#3\textwidth]{#1}
		\caption{#2}\label{fig:#1}
	\end{figure}
	\FloatBarrier
}

% Comando para insertar una imagen sin posición.
% Los parámetros son:
% 1 --> Ruta absoluta/relativa de la figura
% 2 --> Texto a pie de figura
% 3 --> Tamaño en tanto por uno relativo al ancho de página
\newcommand{\imagenflotante}[3]{
	\begin{figure}
		\centering
		\includegraphics[width=#3\textwidth]{#1}
		\caption{#2}\label{fig:#1}
	\end{figure}
}

% El comando \figura nos permite insertar figuras comodamente, y utilizando
% siempre el mismo formato. Los parametros son:
% 1 --> Porcentaje del ancho de página que ocupará la figura (de 0 a 1)
% 2 --> Fichero de la imagen
% 3 --> Texto a pie de imagen
% 4 --> Etiqueta (label) para referencias
% 5 --> Opciones que queramos pasarle al \includegraphics
% 6 --> Opciones de posicionamiento a pasarle a \begin{figure}
\newcommand{\figuraConPosicion}[6]{%
  \setlength{\anchoFloat}{#1\textwidth}%
  \addtolength{\anchoFloat}{-4\fboxsep}%
  \setlength{\anchoFigura}{\anchoFloat}%
  \begin{figure}[#6]
    \begin{center}%
      \Ovalbox{%
        \begin{minipage}{\anchoFloat}%
          \begin{center}%
            \includegraphics[width=\anchoFigura,#5]{#2}%
            \caption{#3}%
            \label{#4}%
          \end{center}%
        \end{minipage}
      }%
    \end{center}%
  \end{figure}%
}

%
% Comando para incluir imágenes en formato apaisado (sin marco).
\newcommand{\figuraApaisadaSinMarco}[5]{%
  \begin{figure}%
    \begin{center}%
    \includegraphics[angle=90,height=#1\textheight,#5]{#2}%
    \caption{#3}%
    \label{#4}%
    \end{center}%
  \end{figure}%
}
% Para las tablas
\newcommand{\otoprule}{\midrule [\heavyrulewidth]}
%
% Nuevo comando para tablas pequeñas (menos de una página).
\newcommand{\tablaSmall}[5]{%
 \begin{table}
  \begin{center}
   \rowcolors {2}{gray!35}{}
   \begin{tabular}{#2}
    \toprule
    #4
    \otoprule
    #5
    \bottomrule
   \end{tabular}
   \caption{#1}
   \label{tabla:#3}
  \end{center}
 \end{table}
}

%
% Nuevo comando para tablas pequeñas (menos de una página).
\newcommand{\tablaSmallSinColores}[5]{%
 \begin{table}[H]
  \begin{center}
   \begin{tabular}{#2}
    \toprule
    #4
    \otoprule
    #5
    \bottomrule
   \end{tabular}
   \caption{#1}
   \label{tabla:#3}
  \end{center}
 \end{table}
}

\newcommand{\tablaApaisadaSmall}[5]{%
\begin{landscape}
  \begin{table}
   \begin{center}
    \rowcolors {2}{gray!35}{}
    \begin{tabular}{#2}
     \toprule
     #4
     \otoprule
     #5
     \bottomrule
    \end{tabular}
    \caption{#1}
    \label{tabla:#3}
   \end{center}
  \end{table}
\end{landscape}
}

%
% Nuevo comando para tablas grandes con cabecera y filas alternas coloreadas en gris.
\newcommand{\tabla}[6]{%
  \begin{center}
    \tablefirsthead{
      \toprule
      #5
      \otoprule
    }
    \tablehead{
      \multicolumn{#3}{l}{\small\sl continúa desde la página anterior}\\
      \toprule
      #5
      \otoprule
    }
    \tabletail{
      \hline
      \multicolumn{#3}{r}{\small\sl continúa en la página siguiente}\\
    }
    \tablelasttail{
      \hline
    }
    \bottomcaption{#1}
    \rowcolors {2}{gray!35}{}
    \begin{xtabular}{#2}
      #6
      \bottomrule
    \end{xtabular}
    \label{tabla:#4}
  \end{center}
}

%
% Nuevo comando para tablas grandes con cabecera.
\newcommand{\tablaSinColores}[6]{%
  \begin{center}
    \tablefirsthead{
      \toprule
      #5
      \otoprule
    }
    \tablehead{
      \multicolumn{#3}{l}{\small\sl continúa desde la página anterior}\\
      \toprule
      #5
      \otoprule
    }
    \tabletail{
      \hline
      \multicolumn{#3}{r}{\small\sl continúa en la página siguiente}\\
    }
    \tablelasttail{
      \hline
    }
    \bottomcaption{#1}
    \begin{xtabular}{#2}
      #6
      \bottomrule
    \end{xtabular}
    \label{tabla:#4}
  \end{center}
}

%
% Nuevo comando para tablas grandes sin cabecera.
\newcommand{\tablaSinCabecera}[5]{%
  \begin{center}
    \tablefirsthead{
      \toprule
    }
    \tablehead{
      \multicolumn{#3}{l}{\small\sl continúa desde la página anterior}\\
      \hline
    }
    \tabletail{
      \hline
      \multicolumn{#3}{r}{\small\sl continúa en la página siguiente}\\
    }
    \tablelasttail{
      \hline
    }
    \bottomcaption{#1}
  \begin{xtabular}{#2}
    #5
   \bottomrule
  \end{xtabular}
  \label{tabla:#4}
  \end{center}
}



\definecolor{cgoLight}{HTML}{EEEEEE}
\definecolor{cgoExtralight}{HTML}{FFFFFF}

%
% Nuevo comando para tablas grandes sin cabecera.
\newcommand{\tablaSinCabeceraConBandas}[5]{%
  \begin{center}
    \tablefirsthead{
      \toprule
    }
    \tablehead{
      \multicolumn{#3}{l}{\small\sl continúa desde la página anterior}\\
      \hline
    }
    \tabletail{
      \hline
      \multicolumn{#3}{r}{\small\sl continúa en la página siguiente}\\
    }
    \tablelasttail{
      \hline
    }
    \bottomcaption{#1}
    \rowcolors[]{1}{cgoExtralight}{cgoLight}

  \begin{xtabular}{#2}
    #5
   \bottomrule
  \end{xtabular}
  \label{tabla:#4}
  \end{center}
}



\graphicspath{ {./img/} }

% Capítulos
\chapterstyle{bianchi}
\newcommand{\capitulo}[2]{
	\setcounter{chapter}{#1}
	\setcounter{section}{0}
	\setcounter{figure}{0}
	\setcounter{table}{0}
	\chapter*{\thechapter.\enskip #2}
	\addcontentsline{toc}{chapter}{\thechapter.\enskip #2}
	\markboth{#2}{#2}
}

% Apéndices
\renewcommand{\appendixname}{Apéndice}
\renewcommand*\cftappendixname{\appendixname}

\newcommand{\apendice}[1]{
	%\renewcommand{\thechapter}{A}
	\chapter{#1}
}

\renewcommand*\cftappendixname{\appendixname\ }

% Formato de portada
\makeatletter
\usepackage{xcolor}
\newcommand{\tutor}[1]{\def\@tutor{#1}}
\newcommand{\course}[1]{\def\@course{#1}}
\definecolor{cpardoBox}{HTML}{E6E6FF}
\def\maketitle{
  \null
  \thispagestyle{empty}
  % Cabecera ----------------
\noindent\includegraphics[width=\textwidth]{cabecera}\vspace{1cm}%
  \vfill
  % Título proyecto y escudo informática ----------------
  \colorbox{cpardoBox}{%
    \begin{minipage}{.8\textwidth}
      \vspace{.5cm}\Large
      \begin{center}
      \textbf{TFG del Grado en Ingeniería Informática}\vspace{.6cm}\\
      \textbf{\LARGE\@title{}}
      \end{center}
      \vspace{.2cm}
    \end{minipage}

  }%
  \hfill\begin{minipage}{.20\textwidth}
    \includegraphics[width=\textwidth]{escudoInfor}
  \end{minipage}
  \vfill
  % Datos de alumno, curso y tutores ------------------
  \begin{center}%
  {%
    \noindent\LARGE
    Presentado por \@author{}\\ 
    en Universidad de Burgos --- \@date{}\\
    Tutor: \@tutor{}\\
  }%
  \end{center}%
  \null
  \cleardoublepage
  }
\makeatother

\newcommand{\nombre}{Fernando Pisot Serrano}


% Datos de portada
\title{\fontsize{14pt}{16pt}\selectfont Eco City Tours: Aplicación móvil para la generación de rutas turísticas sostenibles propuestas por modelos de lenguaje a gran escala}
\author{\nombre}
\tutor{Carlos López Nozal}
\date{\today}

\begin{document}

\maketitle


\newpage\null\thispagestyle{empty}\newpage


%%%%%%%%%%%%%%%%%%%%%%%%%%%%%%%%%%%%%%%%%%%%%%%%%%%%%%%%%%%%%%%%%%%%%%%%%%%%%%%%%%%%%%%%
\thispagestyle{empty}


\noindent\includegraphics[width=\textwidth]{cabecera}\vspace{1cm}

\noindent D. Carlos López Nozal, profesor del departamento de Ingeniería Informática, área de Lenguajes y Sistemas Informáticos.

\noindent Expone:

\noindent Que el alumno D. \nombre, con DNI 70873328R, ha realizado el Trabajo final de Grado en Ingeniería Informática. 

\noindent Y que dicho trabajo ha sido realizado por el alumno bajo la dirección del que suscribe, en virtud de lo cual se autoriza su presentación y defensa.

\begin{center} %\large
En Burgos, {\large \today}
\end{center}

\vfill\vfill\vfill

\begin{center}
  Vº. Bº. del Tutor:\\[2cm]
  D. Carlos López Nozal \@tutor{}
  \end{center}


\newpage\null\thispagestyle{empty}\newpage




\frontmatter

% Abstract en castellano
\renewcommand*\abstractname{Resumen}
\begin{abstract}
Eco City Tours es una aplicación móvil desarrollada con Flutter que propone al usuario rutas turísticas sostenibles. La aplicación se enfoca en la promoción de rutas no motorizadas, optimizadas para ciclistas y peatones, que conectan lugares de interés con el objetivo de fomentar la movilidad sostenible. Las rutas se generan con tecnologías usando \acrfull{gis}. Los \acrfull{pdi} se enriquecen con información detallada sobre lugares turísticos mediante inteligencia artificial consultando en lenguaje natural a \acrfull{llm}.

Esta aplicación se alinea con el concepto de \acrshort{sc}, promoviendo activamente los \acrfull{ods}, con un enfoque particular en el ODS11 (Ciudades y Comunidades Sostenibles).


\end{abstract}

\renewcommand*\abstractname{Descriptores}
\begin{abstract}
Movilidad Sostenible, ODS, LLM, Smart City, Flutter.
\end{abstract}

\clearpage

% Abstract en inglés
\renewcommand*\abstractname{Abstract}
\begin{abstract}
	Eco City Tours is a mobile application \textbf{developed with Flutter} that suggests tourist routes to users. The application focuses on promoting non-motorized routes, optimized for cyclists and pedestrians, connecting points of interest with the goal of fostering sustainable mobility. The routes are generated using technologies based on \acrfull{sig}. The \acrfull{poie} are enriched with detailed information about tourist spots through artificial intelligence, consulting \acrfull{llme} via natural language queries.
	
	This application aligns with the concept of \acrshort{sc}, actively promoting the \acrfull{sdg}, with a particular focus on the SDG11 (Sustainable Cities and Communities).
\end{abstract}

\renewcommand*\abstractname{Keywords}
\begin{abstract}
	Sustainable Mobility, SDGs, LLM, Smart City, Flutter.
\end{abstract}



\clearpage

% Indices
\tableofcontents

\clearpage

\listoffigures

\clearpage

\listoftables
\clearpage

\mainmatter
\capitulo{1}{Introducción}


El crecimiento de población en las ciudades CITA, el turismo como catalizador de la gentrificación supone el gran caballo de batalla de los ayuntamientos y gobiernos locales que han visto como la falta de legislación y control del turismo supone un grave problema agravado aún más si cabe por el malestar de las comunidades locales. La deriva de este problema de difícil solución se centra en una necesidad de desarrollar ciudades más sostenibles e inclusivas para evitar este malestar de las comunidades y su desarrollo incontrolado. A pesar de los avances en la promoción de un nuevo modelo, muchas urbes aún enfrentan desafíos significativos en la integración de prácticas sostenibles en la vida cotidiana de sus habitantes. La falta de información accesible y personalizada sobre rutas y actividades que promuevan la movilidad sostenible y el turismo responsable es un marco común que se debe desarrollar si se quiere evitar que el conflicto crezca sin fin. Esta brecha de información impide que tanto residentes como turistas adopten hábitos más sostenibles que beneficien a la comunidad local y al medio ambiente en un marco global.

A partir del año 2000, muchos gobiernos empezaron a desarrollar soluciones TIC que gracias a fondos económicos europeos promovieron el termino \acrfull{sdg11}, donde marcos de trabajo intentaban fomentar 

En este contexto, la aplicación móvil que proponemos se alinea con estos esfuerzos al proporcionar una herramienta práctica y accesible para la promoción del \acrfull{ods11} y la movilidad sostenible.

Este \acrfull{ods}, se centra en hacer que las ciudades y los asentamientos humanos sean sostenibles y emerge como un pilar fundamental de desarrollo si la población en los centros de las ciudades sigue creciendo. Según el informe de síntesis de la UNESCO sobre el ODS 11, este objetivo no solo es crucial por sí mismo, sino que actúa como un multiplicador, influyendo indirectamente en la consecución de otros \acrshort{ods} debido a su enfoque integral y transversal\footnote{\href{https://uis.unesco.org/sites/default/files/documents/sdg_11_synthesis_report_2023_v11_0_4.pdf}{UNESCO. (2023). SDG 11 Synthesis Report.}}.

La aplicación producto de este \acrshort{tfg}, desarrollada con Flutter, utiliza modelos de lenguaje de gran escala (\acrfull{llm}) y el marco de trabajo LangChain para generar rutas turísticas personalizadas. Estas rutas conectan puntos de interés (\acrfull{poi}) y se visualizan mediante herramientas open-source como \acrfull{osm}.

La aplicación se enfoca en las preferencias del usuario, ofreciendo rutas optimizadas para ciclistas y peatones, y promoviendo así la movilidad sostenible. Al integrar datos y tecnología avanzada, nuestra solución no solo facilita una experiencia turística enriquecedora, sino que también fomenta prácticas sostenibles que pueden tener un impacto positivo en la comunidad y el medio ambiente.

De esta manera se favorece el desarrollo sostenible de las ciudades..........

\end{document}
\capitulo{2}{Objetivos del proyecto}

\section{Objetivos Funcionales}

Estos objetivos se centran en las funcionalidades y características que debe tener la aplicación para satisfacer las necesidades y expectativas de los usuarios. A continuación se detallan los objetivos funcionales del proyecto:

\begin{itemize}
    \item \textbf{Propuesta de Rutas Turísticas Personalizadas}: La aplicación debe ser capaz de generar rutas turísticas personalizadas basadas en las preferencias del usuario utilizando modelos de lenguaje de gran escala (\acrfull{llm}) y el framework \textbf{LangChain}.
    \item \textbf{Obtener los Puntos de Interés (\acrfull{poi})}: La aplicación debe identificar y conectar diversos puntos de interés, proporcionando información relevante sobre cada uno.
    \item \textbf{Visualización de Rutas en Mapa}: La aplicación debe mostrar las rutas sugeridas en un mapa utilizando herramientas libres como \acrfull{osm}.
    \item \textbf{Optimización para Ciclistas y Peatones}: La aplicación debe promover la movilidad sostenible sugiriendo rutas optimizadas para ciclistas y peatones.
    \item \textbf{Interfaz Intuitiva y Amigable}: El usuario debe interactuar con la aplicación de manera intuitiva, siendo fácil de usar por los usuarios las diferentes funcionalidades.
\end{itemize}

\section{Objetivos Técnicos}

Los objetivos técnicos se refieren a los desafíos y metas técnicas que se deben abordar para desarrollar el software. Estos objetivos abarcan aspectos como la arquitectura del sistema, las tecnologías a utilizar y las metodologías de desarrollo. A continuación se detallan los objetivos técnicos del proyecto:

\begin{itemize}
    \item \textbf{Implementación de \acrfull{llm} y \textbf{LangChain}}: Integrar modelos de lenguaje a gran escala (\acrshort{llm}) y el framework \textbf{LangChain} para la generación de rutas y procesamiento de información relevante y ser capaz de integrar dicho conocimiento para ser mostrada en la aplicación móvil así como en un prototipo que muestre de manera incremental la mejora obtenida por parte de los modelos usando diferentes técnicas a la hora de interactuar con ellos como puede ser el uso de técnicas \acrfull{rag}, agentes, etc.
    \item \textbf{Uso de Herramientas Open-Source}: Emplear medios abiertos, libres y gratuitos como \acrfull{osm} para la visualización de mapas y rutas en vez de utilizar servicios que puedan incurrir en gastos para el usuario. 
\end{itemize}

\section{Objetivos Personales}

\begin{itemize}
	\item \textbf{Conocimiento avanzado en \acrshort{llm}}: dada la evolución de esta tecnología, los amplios campos en los que se puede utilizar, obtener una base de conocimientos sería un objetivo que me permitiría expandir mi futuro académico y por tanto distinguir mi perfil profesional especializándome en este sector que se encuentra en fuerte expansión.
	\item \textbf{Creación de aplicación móvil profesional}: de igual manera poner en práctica lo aprendido en varios cursos de \textbf{Dart y Flutter} puede contribuir a que la aplicación de este proyecto sea parte de mi porfolio con aplicaciones que muestren mis habilidades a futuros empleadores.
	\item \textbf{Consecución del \acrshort{tfg} y conclusión de Grado}: al no haber terminado la Ingeniería Técnica Informática por no haber realizado un Proyecto Fin de Carrera en mi pasado, la consecución de este \acrshort{tfg} sirve para convertirme en ingeniero al ser la última asignatura del Grado y supone la consecución de una carga personal de casi veinte años.
\end{itemize}

\capitulo{3}{Conceptos teóricos}

En este capítulo se llevará a cabo una descripción de los conceptos necesarios para comprender el funcionamiento de la aplicación desarrollada.

\section{Conceptos acerca de los Sistemas de Información Geográfica (SIG)}
	\subsection{Navigation service}
	\subsection{Places service y geocoding}
	\section{Conceptos acerca de los Objetivos de Desarrollo Sostenible}
	\subsection{Introducción y contexto histórico}


\section{Conceptos acerca de los modelos de lenguaje a gran escala (LLM)}

Empezamos por el concepto más general para luego ir acercándonos a la parte más concreta del desarrollo de la aplicación. Los modelos de lenguaje a gran escala es un tipo de inteligencia artificial que ha sido entrenada para comprender \acrfull{nlp} que es la manera en que se comunican las personas. Estas inteligencias artificiales son entrenadas entonces con ingentes cantidades de datos que los hacen capaces de comprender peticiones, responder a las mismas en los mismos términos de lenguaje generando una especie de comunicación entre el usuario y la máquina.



\subsection{Uso de LLMs en la aplicación de este TFG}

En este trabajo el uso de los modelos de gran escala han sido usados para obtener los \acrfull{pdi}, basado en un juego de conversaciones con la inteligencia artificial el usuario determina basandose en el conocimiento del modelo qué lugares debería visitar a la hora de hacer turismo sostenible.
En la sección de prototipos de este trabajo se observa como se va construyendo una comunicación con diferentes modelos: desde una conversación básica con resultados mediocres o incluso alucinados, hasta construcciones que tienen en cuenta estructuras de datos que serán construidas como respuesta del modelo al usuario. La aplicación se beneficia de todo ello y genera una respuesta acorde al código que se quiere obtener en la aplicación móvil.
\subsection{Técnicas usadas en los prototipos}
\subsubsection{Zero-shot y Few-shot learning}
\textbf{Zero-shot} se trata de una técnica en la que el usuario no facilita al modelo ningún ejemplo de cómo realizar una tarea. El \acrshort{llm} por tanto interpreta basado en el contexto y su propio entrenamiento lo que se ha requerido y responde acorde a estos datos. Esta técnica se usa cuando lo que se prioriza es la rapidez del modelo frente a la precisión de la salida aportada. Cuando se requiere un trabajo de aproximación mayor una técnica que siempre mejora la conversación con el modelo es la técnica \textbf{few-shot learning}: se facilita en el prompt al modelo unos ejemplos de lo que se quiere obtener. Para comprenderlo mejor veamos el siguiente ejemplo de prompt: 
\begin{verbatim}
	Clasifica los siguientes comentarios como Positivos, 
	Negativos o Neutros:
	
	1. "El producto llegó a tiempo y en perfectas condiciones."
	Clasificación: Positivo
	
	2. "El artículo no cumplió con mis expectativas, 
	estoy decepcionado."
	Clasificación: Negativo
	
	3. "La atención al cliente fue aceptable, pero podría mejorar."
	Clasificación: Neutro
	
	4. "El servicio fue excelente, muy recomendable."
	Clasificación:
\end{verbatim}

Al facilitar tres ejemplos de lo que se quiere obtener, la salida obtenida mejora y es lo que se espera por parte del usuario. Expresar en lenguaje natural lo que se quiere obtener es a veces más difícil y se puede malinterpretar por parte del modelo que darle unos ejemplos para que sepa con precisión el contexto. Más información al respecto se pueden observar en el prototipado del proyecto. Para terminar de ajustar la salida obtenida se usa la siguiente técnica:

\subsubsection{Tool calling o function calling}
Cuando la información del modelo tiene que ser muy precisa se recurre a esta técnica. En el caso del trabajo la información tenía que ser basada en una estructura que desde la programación se pudiera procesar fácilmente. Un archivo cuya estructura fuese en forma de json era vital. Para ello se le pide al modelo qué tipo de salida se requiere y para que no hubiese dudas se le facilitan un par de ejemplos. Una vez establecida la forma de la salida, se procede con el prompt de entrada usando la técnica que se quiera o cumpliendo con las especificaciones del módelo en concreto que se esté usando.
De esta manera también se realiza una separación de abstracción que facilita la modularidad del código: se puede cambiar de origen en los datos, es decir, elegir otro modelo \acrshort{llm}, pero la salida del mismo siempre debe cumplir con estos requisitos desde el punto de vista de la programación. Es el mismo caso de abstracción usada en otros lenguajes de programación donde existe un repositorio y una fuente de datos. El programa se nutre de uno dejando el otro para acceder a datos de manera más concreta, donde el cambio de uno deja inalterado el funcionamiento del programa.

\section{Retrieval-Augmented Generation (RAG)}
Generación Aumentada por Recuperación es una técnica usada en modelos de inteligencia artificial en la cual se obtiene información para nutrir a un modelo de gran escala que ya ha sido entrenado, de esta manera amplía su conocimiento y es capaz de generar una respuesta más precisa, actualizada y completa. 
El problema que subyace en los modelos tradicionales es que una vez alimentados con un conjunto de datos, sufren de un aislamiento del mundo que los rodea.

Para prevenir este problema se nutre de información que el usuario facilita siguiendo los siguientes pasos:
\begin{enumerate}
	\item \textbf{Splitter/tokenización}: la información proporcionada se mide en tokens y cada modelo tiene una cantidad que puede usar como contexto, además del coste que algunos modelos pueden cobrar al usuario por token, es por ello que transformar una cadena de texto inicial que ocupa más espacio del estrictamente necesario en una cadena separada en pequeños trozos de información que además usa ciertos tokens especiales para mayor comprensión es una tarea previa a la recuperación de información.

	\item \textbf{Embeddings}: consiste en transformar la información facilitada y representarla en vectores de n dimensiones. Para ellos se usa comúnmente otro modelo entrenado para transformar la información en vectores.
	
	\imagen{langflow_rag_embeddings}{Preparación de la información de un RAG mostrada en la herramienta Langflow}{1}
	
	\item \textbf{RAG}: con la información ampliada ya vectorizada en una base de datos, el usuario genera una entrada o prompt al modelo, el \acrshort{llm} entonces selecciona la información más afín de los datos aportados para generar así un prompt ampliado o mejorado que será pasado al modelo para un procesamiento de información habitual, consiguiendo así un mejor resultado.
	
	\imagen{langflow_rag_retrieval}{RAG mostrado en la herramienta Langflow usando https://astra.datastax.com }{1}
\end{enumerate}

\subsubsection{Uso de RAGs en la aplicación de este TFG}
La utilidad de los \acrfull{rag} en aplicaciones es muy amplia. La más habitual se usa para conseguir un chatbot de empresas que sirvan como atención al cliente. 
En nuestro caso se alimenta a la base de datos con embeddings la información actualizada de la web usando agentes que serán explicados a continuación, esta información funciona como una entrada de datos de un sistema RAG para la mayor comprensión del mundo que le rodea al modelo. De esta manera y con un juego de prompts \textbf{se obtienen los mejores resultados posibles} que serán después tratados por la aplicación móvil para mostrar dicha información al usuario.


\section{Agentes}
\label{sec:agentes}
La información que alimenta a los RAG puede ser un archivo de texto con información general de un tema sin embargo hay veces en los que la información no está físicamente en un archivo y se tiene que obtener a través de agentes.
Estas múltiples herramientas pueden ser vistas como aplicaciones que alimentarán al modelo con un conjunto de herramientas tales como motores de búsqueda, bases de datos, páginas web, etc. Una vez provisto con esta información el modelo es capaz de razonar acerca de las acciones que debe cumplir para obtener el mejor resultado.
\subsubsection{Uso de Agentes en este TFG}
Se utilizan varios con el fin de obtener a través de la web información actualizada de los puntos de interés de los lugares que se van a visitar.

\section{Conceptos acerca de la Agilidad y el método SCRUM}

\capitulo{4}{Técnicas y herramientas}

El siguiente capítulo presenta las técnicas y herramientas usadas a lo largo del desarrollo de la aplicación Eco City Tours. Se detallarán los aspectos más destacados de cada una de ellas, justificando el porqué de su utilización sobre otras alternativas si las hubiera. Se ha seguido una división en función de su 


\section{Desarrollo relacionado con LLM}
	\subsection{LM Studio}
	LM Studio es una aplicación enfocada en el despliegue de modelos de lenguaje. Su principal objetivo es facilitar la experimentación con \acrlong{llm} ofreciendo un entorno completo y que integra funcionalidades en el procesamiento de lenguaje natural.
	Estas son algunas de las más destacadas:
	
	
	\begin{itemize}
		
		\item \textbf{Búsqueda y despliegue de modelos en local}: a través de un buscador intuitivo, los usuarios pueden seleccionar, descargar y desplegar cualquier modelo disponible en plataformas como Hugging Face, facilitando su uso local.
		
		\item \textbf{Servidor local de un el modelo}: LM Studio permite crear un servidor local que gestione peticiones de información. Esto es especialmente útil para automatizar procesos que requieran el procesamiento de lenguaje o la gestión de grandes volúmenes de datos.
		
		\item \textbf{Entrenamiento y ajuste de modelo}: LM Studio simplifica el proceso de fine-tuning o ajuste fino, que consiste en optimizar un modelo pre-entrenado mediante conjuntos de datos específicos y tareas concretas para mejorar su rendimiento. Gracias a su interfaz gráfica, esta tarea se vuelve más accesible para los desarrolladores, eliminando gran parte de la complejidad técnica.
		
		\item \textbf{Optimización y monitorización}: permite observar el rendimiento de los modelos analizando su consumo de GPU, memoria o CPU así como su tiempo de respuesta, lo que permite realizar al usuario ajustes con el fin de mejorar el rendimiento del sistema.
		
	\end{itemize}
	En un punto inicial del desarrollo se usaba Ollama para experimentar, sin embargo
	La experimentación con diferentes modelos a la hora de decidir cual usar para Eco City Tours fue mucho más rápida y eficiente con esta herramienta.
	
	\subsection{LangChain}
	En el último año LangChain se ha establecido como uno de los marcos de trabajo más populares del mercado. Esta herramienta multiusos aúna aplicaciones tan necesarias para el mundo de los \acrfull{llm} como pueden ser base de datos de vectores, memoria, prompts, herramientas, agentes como ya hemos visto en la sección \ref{sec:agentes} y cadenas de pensamiento (de así su nombre chain). En el prototipo de prompting de este \acrshort{tfg} se puede ver el anidamiento de componentes como son estas cadenas para obtener la mejor entrada posible al modelo y obtener la mejor salida posible, estas cadenas pueden unir componentes como prompts, retrievers, processors, tools o incluso otras cadenas para procesos más complejos.
	Con todo ello LangChain supone una manera de combinar el poder de los \acrshort{llm} con la lógica de cualquier aplicación.
	
	\subsection{LangFlow}
	LangFlow es una herramienta de código abierto que proporciona una \acrfull{gui}, permitiendo la interacción con \acrlong{llm} (Modelos de Lenguaje a Gran Escala) sin necesidad de escribir código. Los usuarios pueden configurar componentes y módulos de forma visual, conectándolos entre sí como si se tratara de un diagrama de flujo, facilitando la creación de aplicaciones complejas de procesamiento de lenguaje.
	
	La aplicación se puede instalar localmente, donde se ejecuta como un servicio en un puerto específico del localhost, permitiendo interactuar con la interfaz a través de un navegador web, como si fuera una aplicación web estándar.
	\begin{itemize}
		
		\item \textbf{Integración con múltiples LLM y herramientas de terceros}: existe una gran variedad de modelos con los que se puede interactuar con solo configurar sus claves API, pero también herramientas como agentes o tokenizadores, lo que amplía las posibilidades de personalización y experimentación.
		
		\item \textbf{Herramienta de exportación e importación}: la plataforma permite exportar el flujo de trabajo generado a un archivo JSON, lo que facilita la portabilidad del proyecto.
		
		\item \textbf{Prototipos iniciales}: se puede encontrar en el arranque de la aplicación proyectos por defecto que facilitan el desarrollo. Por ejemplo, si se quiere conseguir un sistema RAG, se puede partir de una plantilla ya creada que permite ahorrar tiempo a la hora de personalizar un prototipo personal.
	
	\end{itemize}
	En resumen, la principal característica que hace a esta herramienta tan potente es que favorece la experimentación y configuración personal de los elementos que componen un prototipo que usa interacciones con modelos \acrshort{llm}. Esto la convierte en una herramienta ideal para desarrolladores que están empezando a explorar el mundo de los \acrshort{llm}, ya que su enfoque visual ofrece una ventaja significativa frente a alternativas más técnicas, como los cuadernos Jupyter. Además, en el futuro próximo se presenta como una herramienta docente donde presentar a los alumnos los \acrlong{llm} llevando su utilización a campos como el tratamiento de la información, machine-learning o automatización de procesos.

\section{Desarrollo aplicación móvil}
Se utilizó el framework Flutter para desarrollar Eco City Tours. El entorno de desarrollo elegido fue Visual Studio Code, que ofrece una amplia gama de extensiones y herramientas que facilitan el proceso de desarrollo.
	
	\subsection{Bloc como gestor de estados}
	Un aspecto fundamental en una aplicación móvil reactiva es la gestión del estado, cuyo propósito es automatizar la actualización de las vistas cuando los valores de la lógica de control cambian. Flutter ofrece diversos gestores de estado, como \textit{Provider, Cubit, GetX}, entre otros. Sin embargo, Bloc destaca como un paquete \textbf{Flutter Favorite}, lo que lo sitúa como una opción preferida debido a su soporte activo, calidad de código, seguridad y frecuencia de actualizaciones.
	
	Entre todos los gestores aprendidos durante mi formación, Bloc resultó ser una opción muy robusta. Una vez comprendida su sintaxis, se aprecia su facilidad de uso y la modularidad de sus elementos, que se estructuran en eventos, estados y definiciones de blocs.
	\begin{itemize}
		
		\item \textbf{Gestión clara del estado}: Bloc proporciona una manera estructurada de gestionar los diferentes estados de la aplicación. Cada cambio de estado es manejado a través de eventos, lo que permite una clara separación entre la lógica de negocio y la presentación visual.
		
		\item \textbf{Escalabilidad y mantenibilidad}: con las extensiones de Visual Studio Code, es posible generar nuevos blocs fácilmente, lo que facilita la creación de nuevas funcionalidades sin añadir complejidad innecesaria al código. Cada nueva funcionalidad de la aplicación puede estar controlada por un bloc independiente, lo que resulta en un código escalable y altamente mantenible.
				
		\item \textbf{Reutilización de lógica}: Uno de los beneficios clave de Bloc es que la lógica de negocio se puede reutilizar fácilmente en diferentes partes de la aplicación. Esto facilita la implementación de componentes que comparten comportamientos similares sin duplicar código, simplemente creando Widgets como BlocProvider o MultiBlocProvider como se hace al nivel más alto de una aplicación como se puede ver en el siguiente árbol de Widgets:
		
		\imagen{widget_tree}{Árbol de widgets de Eco City Tours durante su desarrollo}{0.8}
		
				
	\end{itemize}
	
	\subsection{Extensiones de Visual Studio Code más destacadas}
		Se citan a continuación brevemente algunas de las extensiones que facilitaron el desarrollo de la aplicación aumentando la producción:
		\begin{itemize}
		
		\item \textbf{ChatBot}: con un perfil de estudiante de GitHub hay asociados una serie de privilegios entre los que destacan el uso gratuito de esta herramienta que permite tener autocompletado de código y comentarios o un chat con la IA que facilita la depuración de código. 
		
		\item \textbf{Pubspec Assist}: las dependencias de librerías incluidas en el archivo pub\_spec.yaml son automáticamente instaladas, ordenadas y gestionadas en definitiva por este asistente que ahorra múltiples comandos en consola mejorando la rápidez a la hora de programar.
		
		\item \textbf{Snippets}: de manera análoga los snippets de flutter permiten programar rápidamente estructuras del código repetitiva. Por dar un ejemplo si se escribe stlessw y se pulsa tabulador el código de un Stateless Widget aparece escrito en pantalla. De igual manera se pueden personalizar atajos que aumenten la producción del tiempo de desarrollo.
		
	\end{itemize}
		

\section{Gestión de proyectos}
	\subsection{GitHub}
	GitHub es una plataforma en la nube para el alojamiento y gestión de código fuente, que se basa en el sistema de control de versiones Git. A lo largo de la carrera de Ingeniería Informática, ha sido utilizada en diversas asignaturas, lo que la convierte en una herramienta fundamental para el desarrollo de proyectos. En el contexto de este \acrshort{tfg}, GitHub ha sido esencial para centralizar y gestionar el código generado en varios \acrshort{ide}, permitiendo un control de versiones eficiente, la conservación del trabajo y una colaboración estructurada en la nube. Algunas de sus características principales usadas en el desarrollo de Eco City Tours fueron las siguientes: 
	
	\begin{itemize}
		
		\item \textbf{Control de versiones con Git}: gracias a los commits realizados a lo largo del desarrollo se permite comprobar la evolución de un proyecto así como volver a estados del trabajo gestionados en sus ramas. Otras herramientas como pull-requests permiten solicitar cambios al código facilitando un uso colaborativo durante la gestión del proyecto.
		
		\item \textbf{Integración con herramientas de terceros}: GitHub se integra fácilmente con una amplia variedad de herramientas de desarrollo, como servicios de CI/CD, plataformas de despliegue y gestores de proyectos como Zube, lo que permite automatizar tareas y mejorar el flujo de trabajo.
		
		\item \textbf{Documentación y wikis}: GitHub permite mantener documentación clara y estructurada directamente en el repositorio, facilitando la creación de archivos README para proporcionar información detallada sobre el proyecto.
		
	\end{itemize}
	Al trabajar desde diferentes equipos GitHub fue especialmente útil en el desarrollo del proyecto: se utilizó la integración de Visual Studio Code con Github para los cambios en la programación de la aplicación y los prototipos en cuadernos jupyter y se usó la extensión Git Graph para comprobar el funcionamiento y navegación a través de los commits cuando fue necesario. Para realizar commits en los cambios de la documentación se usó GitHub Desktop una vez realizados los cambios con la \acrshort{ide} TeXstudio.
		\imagen{git_graph}{Uso de GitHub con la extensión de VSCode Git Graph}{1}
	
	\subsection{Zube}
	Zube es una plataforma de gestión de proyectos con un enfoque colaborativo durante el seguimiento del desarrollo del mismo. Las características principales que lo hacen la herramienta usada son:
	\begin{itemize}
		
		\item Versión gratuita: Zube ofrece un plan gratuito que cubre las necesidades básicas de gestión de proyectos autogestionados.
		
		\item Conexión con GitHub: su integración con el repositorio del proyecto para su seguimiento es una condición indispensable. Además, La sincronización es inmediata lo que produce por ejemplo que al crear una tarea, inmediatamente esté disponible para su gestión desde Zube.
		
		\item Tableros Kanban: se facilita la visualización del flujo de trabajo. Las tarjetas de tareas pueden moverse fácilmente entre columnas como "Pendiente", "En progreso" y "Completado", lo que permite un seguimiento claro y eficaz del estado de cada tarea..
		
		\item Herramientas gráficas para control de los sprints: otra manera de consultar el flujo de trabajo durante los sprints con gráficos burndown, burnup, throughput o de velocidad.
	 
	\end{itemize}
	
	El uso de esta aplicación fue vital a lo largo del proyecto, ya que se utilizó principalmente para el control de sprints y sus tareas en el panel Kanban y la asignación con puntos de historia. La integración con GitHub permitió un flujo de trabajo eficiente, manteniendo el código fuente y la gestión de tareas en perfecta sincronización, lo que favoreció un desarrollo ágil y organizado. 
	
	\subsection{TeXstudio}
	Para la elaboración de la documentación de este \acrshort{tfg}, se ha optado por utilizar TeXstudio, un \acrfull{ide} especializado en la edición de textos en LaTeX. Esta herramienta facilita la redacción de trabajos académicos y científicos mediante características como el resaltado de sintaxis, corrección ortográfica y semántica en tiempo real, y la posibilidad de compilar el documento durante su edición, lo que permite visualizar el resultado final en formato PDF de manera inmediata.
	
	TeXstudio es una herramienta multiplataforma, y se seleccionó por encima de otras alternativas debido a su facilidad de uso, interfaz intuitiva, y la consola de errores integrada, que simplifica la detección de fallos en la secuencia de comandos LaTeX.
	
	Una de las características más destacadas es la funcionalidad de \textbf{compilación automática al editar}, la cual permite previsualizar continuamente el documento mientras se trabaja, lo que mejora la productividad al ofrecer un resultado inmediato sobre los cambios realizados.
	
	
	
	


\capitulo{5}{Aspectos relevantes del desarrollo del proyecto}

Durante el desarrollo de un \acrlong{tfg}, es inevitable tomar decisiones que impactan de manera significativa en el resultado final. En este capítulo se describen las decisiones clave que han influido en la evolución y estado final de Eco City Tours. La justificación de estas decisiones se espera que pueda servir como una referencia útil para futuros compañeros o desarrolladores que enfrenten objetivos similares. 

\section{Personalización de LLM}
A la hora de obtener información que procesar por el método \acrshort{rag} se valoraron muchas fuentes de datos. Uno de los origenes de datos de información turística favoritos de los usuarios es Tripadvisor. Contar con la información actualizada de este gigante turístico suponía un gran aliciente. Sin embargo se desestimó su uso por varias razones: la información se podía obtener a través del método scraping o webscraping que toma la información en bruto de la página web y se podía postprocesar. Dicha práctica incumpliría los Términos de Uso del Servicio, ya que Tripadvisor usa un acceso a través de API para obtener la información de su base de datos. En primer lugar se necesita una forma de pago para poder empezar a usar el servicio y su utilización si sobrepasa las 5.000 peticiones al mes incurriría en gastos al desarrollador. En este caso se buscaron alternativas que funcionaran de manera análoga a Tripadvisor para nutrir el \acrshort{rag}. \todo{INDICAR CUÁLES CUANDO SE HAGA}

\section{Elección de servicios Google sobre tecnología \acrlong{osm}}

Desde un comienzo en el proyecto se quería utilizar \textbf{código abierto}, pues su filosofía se alinea mejor con los valores aprendidos en la Universidad, donde se promociona el uso de herramientas que no supongan un coste para el alumno, se fomenta su uso evitando la posible discriminación económica y una forma de trabajar colaborativa.

De manera renuente se decide cambiar los servicios necesarios para la visualización y gestión de marcadores y rutas a los establecidos por Google. Los motivos que propiciaron este gran cambio fueron los siguientes:
\begin{itemize}
	\item \textbf{Soporte de un gigante tecnológico}: las herramientas de código abierto aunque algunas tienen un gran seguimiento por la comunidad no pueden competir con la documentación, ejemplos de desarrolladores y tecnología de uso de una potencia como Google.
	\item \textbf{Integración}: Flutter forma parte de Google, lo que supone una integración nativa que hace de su funcionamiento y robustez una de las herramientas usadas.
	\item \textbf{Riesgos asociados a complementos de terceros}: Durante el desarrollo, se exploraron soluciones como \acrfull{osrm} para gestionar rutas, pero estas requerían procesar manualmente las conexiones con el servicio o confiar en paquetes de terceros. Estos paquetes, aunque facilitan el desarrollo, presentan un riesgo mayor debido a su posible discontinuidad o incompatibilidad con futuras actualizaciones. En cambio, el uso de herramientas como Dio \cite{dio_package}, un paquete marcado como favorito por Flutter, garantiza un soporte nativo y más estable en el ecosistema de Google.
\end{itemize} 


\section{Elección de servicios Geocoding MapBox sobre servicios Google}
Los servicios de geocoding son herramientas que permiten convertir direcciones físicas (como ``Calle Mayor, Ciudad, País'') en coordenadas geográficas (latitud y longitud), y viceversa. Esto es esencial para aplicaciones que requieren localización geográfica precisa, como mapas interactivos, planificación de rutas, análisis espacial, o cualquier función que dependa de ubicaciones específicas y los puntos de interés alrededor de una ubicación. Estos servicios en Google tienen el nombre de \textit{Google Places} y su uso supone un coste económico para el desarrollador, incluso con el mínimo tráfico posible.

Para una versión inicial de la aplicación de este \acrshort{tfg} se decidió usar MapBox\cite{mapbox}, una empresa de mapas con un soporte similar a Google pero que no incurre gastos en volúmenes de tráfico como en los que se incurre durante la etapa de desarrollo.
Con Eco City Tours ya publicada y si esta alcanza un uso elevado de peticiones siempre se puede cambiar a \textit{Google Places} ya que la implementación como pasaba con el resto de servicios es nativa y por tanto muy sencilla de llevar a cabo, si se quiere conseguir unificación de servicios y costes. Desde el punto de vista de la programación solo cambia la manera de acceder a los datos pues las coordenadas se presentan primero con longitud y después con latitud, es decir, de manera inversa a los servicios Google. La modularidad del código realizado y el tratamiento de interceptores hace que un futuro cambio de servicios y peticiones GET sean facilmente implementados.

\section{Elección de google generative ai como primer LLM}
Durante el desarrollo se estudiaron múltiples opciones a la hora de obtener la información desde un \acrshort{llm}. 
Siguiendo con el ecosistema de Google

\section{Descarte de módelo en local}
Aunque en un principio el ser capaz de probar la aplicación de una manera 

\capitulo{6}{Trabajos relacionados}

A continuación, se comparan aplicaciones de referencia o similares usadas por usuarios que puedan ser objetivos.

\subsection{Tripadvisor}
\subsection{Wanderlog}
\subsection{Visit A City}
\subsection{Tiqets - Museos y Atracciones}

\subsection{Chatbot - https://github.com/jrg1013/chatbot}
Trabajo Fin de Carrera de José María Redondo Guerra que tuvo como tutores a José Ignacio Santos Martín y Carlos López Nozal.

Este trabajo fue desarrollado usando \acrfull{llm}.

\subsection{Visualización de las actividades socioculturales en Castilla y León CULTURALCyL}
Trabajo Fin de Carrera de Yanela Lozano Pérez con tutores José Ignacio Santos, Virginia Ahedo y Silvia Díaz.
Esta aplicación mostraba información de eventos culturales.

\begin{table}[h!]
	\centering
	\begin{tabular}{|l|c|c|c|c|c|}
		\hline
		Aplicación & Versión de pago & Recomendaciones tienen en cuenta factores de sostenibilidad & Modificación dinámica de rutas & Intereses de terceros pueden afectar a los resultados & Fuente de datos \\ \hline
		Mi aplicación & No & Sí & Sí & No & LLM(RAG-Agents) \\ \hline
		Tripadvisor & Sí & No & Sí & Sí & Propia \\ \hline
		Wanderlog & Sí/No & Sí/No & GoogleMaps & Sí/No & Usuarios \\ \hline
		Visit A City & Sí/No & Sí/No & Sí/No & Sí/No & Usuarios \\ \hline
		Tiqets & Sí/No & Sí/No & Sí/No & Sí/No & Descripción \\ \hline
	\end{tabular}
	\caption{Comparación de aplicaciones similares}
	\label{tabla:comparativa}
\end{table}

\capitulo{7}{Conclusiones y Líneas de trabajo futuras}
\section*{Conclusiones}
\section*{Líneas de trabajo futuras}
Existen múltiples maneras de expandir y llenar de nuevas funcionalidades a la aplicación llevada a cabo.
Por citar algunas que puedan resultar más útiles al usuario:
\begin{itemize}
    \item \textbf{Gamificación:} Recompensas por rutas completadas o distancia recorrida con un medio ecológico.
    \item \textbf{Ratings:} Valorar las rutas permitiendo la busqueda de los mismos.
    \item \textbf{Mejora en planificador de rutas:} Determinación de la zona de sombra.
\end{itemize}


\bibliographystyle{plain}
\bibliography{bibliografia}

\end{document}
